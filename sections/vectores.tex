\documentclass[../template.tex]{subfiles}

\begin{document}

\section{Vectores}
\dfn{Vector}{Un vector es una lista (conjunto ordenado) de datos o cantidades que no se pueden expresar por una única cantidad. De ahora en adelante hablaremos de vectores euclidianos. Que tienen cantidades numéricas y una dimensión.}
\subsection{Vectores Euclidianos}
Un vector euclidiano tiene una dirección y una magnitud. Este tipo de vectores se denotan de la forma $\vec a$.
\ex{Vector en $\mathbb{R}^2$}{
\begin{wrapfigure}{l}{0pt}
\centering
\begin{tikzpicture}
	\draw (2,0) coordinate (A) -- (0,0) coordinate (B) -- (2,1) coordinate (C) pic ["$\theta$", draw, angle eccentricity=2] {angle};
	\draw[black, ->] (-3,0) -- (3,0);
	\draw[black, ->] (0,-3) -- (0,3);
	\draw[red, -stealth] (0,0) -- (2,1) node[above] {$\vec a = \begin{bmatrix}
	2\\
	1\\
	\end{bmatrix}$};
	\draw[gray, |-|] (0,-0.1) -- (2,-0.1) node[midway, below] {$2$};
	\draw[gray, |-|] (2.1,0) -- (2.1,1) node[midway, right] {$1$};
	
\end{tikzpicture}
\end{wrapfigure}
La magnitud de $\vec a$ se denota de la forma $|\vec a|$, en este caso sería.
\begin{align}
	|\vec a| &= \sqrt{2^2+1^2}\\
		 &= \sqrt{5}
\end{align}
Por geometría la dirección sería
\begin{align}
	\tan{\theta} &= \frac{1}{2}\\
	\theta &= \tan^{-1}{\frac{1}{2}}\\
	       &\approx 0.4636
\end{align}
\vspace{1cm}
}
\subsubsection{Adición}
\dfn{Adición de Vectores Euclidianos}{
La adición de vectores euclidianos se realiza
}

\subsubsection{Substracción}



\subsection{Productos internos}
\dfn{Producto Interno}{
Un producto interno denotado por $\left< \vec x, \vec y \right> $ donde $\vec x$ y $\vec y$ son vectores es una operación que cumple las siguientes propiedades
\begin{itemize}
	\item $\left< \vec x,\vec y \right> = \left< \vec y,\vec x \right>$
	\item $\left< \vec x,\vec x \right> > 0 \mid x \neq \boldmath{0}$
	\item $\left< a\vec x + b\vec z,\vec y \right> = a\left< \vec x, \vec y \right> + b\left<\vec z,\vec y \right>$
\end{itemize}
}
\nt{El producto interno es un operador lineal, después veremos que significa eso.}
\ex{Producto Punto}{
El producto punto denotado por $\cdot$ entre dos vectores es la multiplicación y adición entrada por entrada de ambos vectores. Este también es un producto interno.
\begin{align}
	\vec a \cdot \vec b &= \sum_{i = 1}^{N}a_i b_i 
\end{align}
El producto punto también tiene otras propiedades geométricas importantes
\begin{equation}
	\vec a \cdot \vec b = |\vec a||\vec b| \cos{\theta}
\end{equation}
}
\clm{El producto punto es un producto interno}{}{}
\begin{myproof}{El producto punto es un producto interno}
	\begin{align}
		\intertext{Propiedad 1. }
		\vec a \cdot \vec b &= \vec b \cdot \vec a\\
		\sum_{i = 1}^{N}a_i b_i &= \sum_{i = 1}^{N}b_i a_i\\
		\sum_{i = 1}^{N}a_i b_i &= \sum_{i = 1}^{N}a_i b_i\\
		\intertext{Propiedad 2.}
		\vec a \cdot \vec a &= \sum_{i = 1}^{N}a_i^2\\
		a_i \in \mathbb{R} &\implies a_i^2 \geq 0\\
		\intertext{Propiedad 3.}
		\left( \alpha \vec a + \zeta \vec z \right) \cdot \beta \vec b &= \sum_{i = 1}^{N}\left( \alpha a_i + \zeta z_i \right) \beta b_i\\
		\alpha \vec a \cdot \beta \vec b + \zeta \vec z \cdot \beta \vec b &= \sum_{i = 1}^{N} \alpha a_i \beta b_i + \zeta z_i \beta b_i
	\end{align}
\end{myproof}
\clm{$\vec a \cdot \vec b = |\vec a||\vec b| \cos{\theta}$}{}{}

\end{document}

