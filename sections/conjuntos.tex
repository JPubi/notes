\documentclass[../template.tex]{subfiles}


\begin{document}

\section{Conjuntos}
\dfn{Conjunto}{Un conjunto es una colección de objetos, que llamamos elementos.} De ahora en adelante nos referiremos a los conjuntos con una letra mayúscula ($A,B,C,\ldots $). Mientras que los elementos se escribirán con letras minúsculas ($a_1, a_2, a_3, \ldots, a_n \in A$).
\subsection{Operaciones}
\dfn{Unión}{
La operación de unión se realiza entre conjuntos, y su símbolo es $\cup$.
\[ 
	A \cup B = \left\{x_i \mid x_i \in A \lor x_i \in B \right\}
.\]
}
\dfn{Intersección}{
La operación de intersección se realiza entre conjuntos, y su símbolo es $\cap$
\[ 
	A \cap B = \left\{x_i \mid x_i \in A \land x_i \in B \right\}
.\]
}
\dfn{Cardinalidad}{
Se puede entender la cardinalidad de un conjunto finito, por la cantidad elementos que lo conforman. La cardinalidad es denotada por $|A|$ donde $A$ es un conjunto.}
\ex{Cardinalidad}{
\begin{align}
	A &= \left\{a_1, a_2, a_3, a_4, a_5\right\}\\
	|A| &= 5
\end{align}
}
\dfn{
$|A| = |B|$
}{
	La cardinalidad de $A$ es igual a la cardinalidad de $B$ si existe una función biyectiva entre sus elementos. Es decir si existe una función ($f$) cuya imagen asigne a todos los valores del conjunto $B$ únicamente a un valor en $A$ y vice-versa. Esto también implica que existe una función inversa ($f^{-1}$)
}
\dfn{$|A| \leq |B|$}{
La cardinalidad de $A$ es menor que o igual a la de $B$ si existe una función inyectiva de $A$ a $B$ es decir existe una función que asigna todos los elementos de $A$ a $B$ de forma única pero la imagen de la función no abarca todo el conjunto $B$.
}
\dfn{Conjunto Potencia}{
Denotado por $P(A)$ donde $A$ es un conjunto, este esta compuesto por todos los subconjuntos de $A$ y el vacío.}
\ex{Conjunto Potencia}{
\begin{align} 
	A &= \left\{a_1, a_2, a_3\right\}\\
	P(A) &= \left\{\emptyset,
\left\{a_1\right\},
\left\{a_2\right\},
\left\{a_3\right\},
\left\{a_1,a_2\right\},
\left\{a_2,a_3\right\},
\left\{a_1, a_3\right\},
\left\{a_1, a_2, a_3\right\}
\right\}
\end{align}
}
\nt{
El conjunto poder $P(A)$ tiene una cardinalidad de $2^n$ donde $n$ es la cardinalidad de $A$.
}


\end{document}


