\documentclass{report}

\input{preamble}
\input{macros}
\input{letterfonts}

\title{\Huge{Some Class}\\Random Examples}
\author{\huge{Ethan Ordaz}}
\date{\today}
\usepackage{wrapfig}
\usetikzlibrary{angles, quotes}


% \begin{algorithm}[H]
% \KwIn{This is some input}
% \KwOut{This is some output}
% \SetAlgoLined
% \SetNoFillComment
% \tcc{This is a comment}
% \vspace{3mm}
% some code here\;
% $x \leftarrow 0$\;
% $y \leftarrow 0$\;
% \uIf{$ x > 5$} {
%     x is greater than 5 \tcp*{This is also a comment}
% }
% \Else {
%     x is less than or equal to 5\;
% }
% \ForEach{y in 0..5} {
%     $y \leftarrow y + 1$\;
% }
% \For{$y$ in $0..5$} {
%     $y \leftarrow y - 1$\;
% }
% \While{$x > 5$} {
%     $x \leftarrow x - 1$\;
% }
% \Return Return something here\;
% \caption{what}
% \end{algorithm}

% \ex{Open Set and Close Set}{
% 	\begin{tabular}{rl}
% 		Open Set:   & $\bullet$ $\phi$                                              \\
% 		            & $\bullet$ $\bigcup\limits_{x\in X}B_r(x)$ (Any $r>0$ will do) \\[3mm]
% 		            & $\bullet$ $B_r(x)$ is open                                    \\
% 		Closed Set: & $\bullet$ $X,\ \phi$                                          \\
% 		            & $\bullet$ $\overline{B_r(x)}$                                 \\
% 		            & $x-$axis $\cup$ $y-$axis
% 	\end{tabular}
% }

% \begin{tikzpicture}
% 	\draw[red] (0,0) circle [x radius=3.5cm, y radius=2cm] ;
% 	\draw (3,1.6) node[red]{$V$};
% 	\draw [blue] (1,0) circle (1.45cm) ;
% 	\filldraw[blue] (1,0) circle (1pt) node[anchor=north]{$u$};
% 	\draw (2.9,0.4) node[blue]{$B_r(u)$};
% 	\draw [green!40!black] (1.7,0) circle (0.5cm) node [yshift=0.7cm]{$B_{\delta}(x)$} ;
% 	\filldraw[green!40!black] (1.7,0) circle (1pt) node[anchor=west]{$x$};
% \end{tikzpicture}

% \dfn{}{} make definition environment

% \qs{}{} make question

% \sol add solution

% \nt{} make note

% \clm{}{}{} make claim

% \ex{}{} make example

% \thm{}{} make theorem

% \begin{myproof} make proof env

% \cor{}{} corolary

% \mlenma{}{} lemma

% \mprop{}{} proposition

\begin{document}

\maketitle
\newpage% or \cleardoublepage
% \pdfbookmark[<level>]{<title>}{<dest>}
\pdfbookmark[section]{\contentsname}{toc}
\tableofcontents
\pagebreak

\chapter{Álgebra Lineal}
\section{Conjuntos}
Un conjunto es una colección de objetos, que llamamos elementos. De ahora en adelante nos referiremos a los conjuntos con una letra mayúscula ($A,B,C,\ldots $). Mientras que los elementos se escribirán con letras minúsculas ($a_1, a_2, a_3, \ldots, a_n \in A$).
\subsection{Operaciones}
\subsubsection{Unión}
La operación de unión se realiza entre conjuntos, y su símbolo es $\cup$.
\[ 
	A \cup B = \left\{x_i \mid x_i \in A \lor x_i \in B \right\}
.\]
\subsubsection{Intersección}
La operación de intersección se realiza entre conjuntos, y su símbolo es $\cap$
\[ 
	A \cap B = \left\{x_i \mid x_i \in A \land x_i \in B \right\}
.\]
\subsubsection{Cardinalidad}
Se puede entender la cardinalidad de un conjunto finito, por la cantidad elementos que lo conforman. La cardinalidad es denotada por $|A|$ donde $A$ es un conjunto.
\ex{Cardinalidad}{
\begin{align}
	A &= \left\{a_1, a_2, a_3, a_4, a_5\right\}\\
	|A| &= 5
\end{align}
}
\dfn{
$|A| = |B|$
}{
	La cardinalidad de $A$ es igual a la cardinalidad de $B$ si existe una función biyectiva entre sus elementos. Es decir si existe una función ($f$) cuya imagen asigne a todos los valores del conjunto $B$ únicamente a un valor en $A$ y vice-versa. Esto también implica que existe una función inversa ($f^{-1}$)
}
\dfn{$|A| \leq |B|$}{
La cardinalidad de $A$ es menor que o igual a la de $B$ si existe una función inyectiva de $A$ a $B$ es decir existe una función que asigna todos los elementos de $A$ a $B$ de forma única pero la imagen de la función no abarca todo el conjunto $B$.
}
\subsubsection{Conjunto Potencia}
Denotado por $P(A)$ donde $A$ es un conjunto, este esta compuesto por todos los subconjuntos de $A$ y el vacío.
\begin{align} 
	A &= \left\{a_1, a_2, a_3\right\}\\
	P(A) &= \left\{\emptyset,
\left\{a_1\right\},
\left\{a_2\right\},
\left\{a_3\right\},
\left\{a_1,a_2\right\},
\left\{a_2,a_3\right\},
\left\{a_1, a_3\right\},
\left\{a_1, a_2, a_3\right\}
\right\}
\end{align}
El conjunto poder $P(A)$ tiene una cardinalidad de $2^n$ donde $n$ es la cardinalidad de $A$.

\section{Grupo}
Un grupo es un conjunto con una operación ($f(x,y)$) que satisface algunas condiciones.
\begin{itemize}
\item De un elemento del conjunto se puede llegar al otro con la operación.
\item Todo elemento del conjunto tiene un inverso.
\item Existe un elemento neutro.
\item La operación de dos elementos esta dentro del conjunto ($f(x,y) \in A \forall x,y \in A$)
\end{itemize}
Si la operación es conmutativa se le conoce como un Grupo Abeliano.
\ex{$\mathbb{Z}$ y $+$}{
	$\mathbb{Z}$ y la suma ($f(x,y) = x+y$) son un grupo abeliano.
	\begin{itemize}
	\item Existe un neutro, 0. $f(x,0) = x$
	\item Se puede llegar a todos los números desde uno dado
		\begin{align}
			\intertext{Con $x$ y $z$ dados y $x$ no es el neutro}
			f(x,y) &= z\\
			x+y &= z\\
			y &= z-x
			\intertext{Entonces encontramos un número $y\neq z$ por el cual podemos llegar a cualquier $z$ desde una $x$}
		\end{align}
	\item Todo número tiene su inverso.
	\item La suma es conmutativa.
	\end{itemize}
}
\section{Vectores}
Un vector es una lista (conjunto ordenado) de datos o cantidades que no se pueden expresar por una única cantidad. De ahora en adelante hablaremos de vectores euclidianos. Que tienen cantidades numéricas y una dimensión.
\subsection{Vectores Euclidianos}
Un vector tiene una dirección y una magnitud. Este tipo de vectores se denotan de la forma $\vec a$.
\ex{Vector en $\mathbb{R}^2$}{
\begin{wrapfigure}{l}{0pt}
\centering
\begin{tikzpicture}
	\draw (2,0) coordinate (A) -- (0,0) coordinate (B) -- (2,1) coordinate (C) pic ["$\theta$", draw, angle eccentricity=2] {angle};
	\draw[black, ->] (-3,0) -- (3,0);
	\draw[black, ->] (0,-3) -- (0,3);
	\draw[red, -stealth] (0,0) -- (2,1) node[above] {$\vec a = \begin{bmatrix}
	2\\
	1\\
	\end{bmatrix}$};
	\draw[gray, |-|] (0,-0.1) -- (2,-0.1) node[midway, below] {$2$};
	\draw[gray, |-|] (2.1,0) -- (2.1,1) node[midway, right] {$1$};
	
\end{tikzpicture}
\end{wrapfigure}
La magnitud de $\vec a$ se denota de la forma $|\vec a|$, en este caso sería.
\begin{align}
	|\vec a| &= \sqrt{2^2+1^2}\\
		 &= \sqrt{5}
\end{align}
Por geometría la dirección sería
\begin{align}
	\tan{\theta} &= \frac{1}{2}\\
	\theta &= \tan^{-1}{\frac{1}{2}}\\
	       &\approx 0.4636
\end{align}
\vspace{1cm}
}
\end{document}
